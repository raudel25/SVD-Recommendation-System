\documentclass{article}

\usepackage[utf8]{inputenc} 
	\usepackage[text={18cm,21cm},centering]{geometry}
	\usepackage{amsmath,amssymb,amsfonts,latexsym,cancel} 
	\usepackage[T1]{fontenc} % Comandos personales - especiales 
	\usepackage{titlesec} 
	\usepackage{hyperref}
	\newcommand{\sen}{\mathop{\rm sen}\nolimits} %seno 
	\newcommand{\arcsen}{\mathop{\rm arcsen}\nolimits} 
	\newcommand{\arcsec}{\mathop{\rm arcsec}\nolimits} 
	\newcommand{\R}{\mathbb{R}} \newcommand{\N}{\mathbb{N}} 
	\newcommand{\Z}{\mathbb{Z}} \def\max{\mathop{\mbox{\rm máx}}} % máximo \def\min{\mathop{\mbox{\rm mín}}} % mínimo

% Teoremas
%--------------------------------------------------------------------------
\newtheorem{thm}{Teorema}[section]
\newtheorem{cor}[thm]{Corolario}
\newtheorem{lem}[thm]{Lema}
\newtheorem{prop}[thm]{Proposición}
\newtheorem{defn}[thm]{Definición}
\newtheorem{rem}[thm]{Observación}

% De la misma forma se pueden definir comandos con argumentos. Por
% ejemplo, aquí definimos un comando para escribir el valor absoluto
% de algo más fácilmente.
%--------------------------------------------------------------------------
\newcommand{\abs}[1]{\left\vert#1\right\vert}

% Operadores.
% Los operadores nuevos deben definirse como tales para que aparezcan
% correctamente. Como ejemplo definimos en jacobiano:
%--------------------------------------------------------------------------
\DeclareMathOperator{\Jac}{Jac}

%--------------------------------------------------------------------------
\title{Sistemas de Recomendación basados en SVD}
\author{Daniel Abad Fundora C212\\
		Anabel Benítez González C211\\
		Raudel Alejandro Gómez Molina C211\\
		Alex Sierra Alcalá C211\\\\		
  \small SVD: Siempre venciendo desafios\\
  \small Matemática Numérica\\
  \small Facultad de Matemática y Computación 
}

\begin{document}
\maketitle

\section{Sistemas de Recomendación}

Desde el nacimiento de internet como tecnología se han abierto nuevas posibilidades que facilitan el acceso a la información, también esta se ha aumentado considerablemente. Tal impacto se ha visto reflejado en todo tipo de áreas, una de ellas es el comercio, donde sitios especializados en internet ofertan millones de productos o servicios, también hay sitios donde la 
mayoría de datos pueden ser efímeros o de poca utilidad, produciendo un caos de información que, al contrario de satisfacer requerimientos de los usuarios, pueden complicar más su respuesta; para que sea efectiva, depende de la habilidad y experiencia que tenga el usuario para expresar sus necesidades mediante una consulta y está demostrado que en la mayoría de casos es imprecisa y vaga. Por lo anterior surgen los sistemas de recuperación de información, que si bien apuntan a mejorar la forma en que se almacena, representa y organiza la información, su función no es devolver la información deseada por el usuario sino únicamente indicar qué documentos son potencialmente relevantes para dicha necesidad de información. A partir de esto, surge una nueva necesidad de encontrar herramientas que se encarguen de buscar por los usuarios, que los conozcan y recomienden un conjunto limitado de opciones acordes a sus intereses. 

Un sistema de recomendación se puede definir, de manera formal, como aquel sistema que tiene como principal tarea seleccionar ciertos objetos de acuerdo con los requerimientos del usuario. Estos sistemas son muy atractivos en situaciones donde la cantidad de información que se ofrece al usuario supera ampliamente cualquier capacidad individual de exploración.\\\\
La creación de un sistema de recomendación cuenta con tres fases principales:
\begin{enumerate}
	\item[-] Captura de las preferencias y los gustos e intereses del usuario.
	\item[-] Extracción del conocimiento, aquí el sistema se encarga de interpretar la información que recopiló anteriormente para poder predecir los gustos y las preferencias del usuario.
	\item[-] A partir del contenido procesado anteriormente el sistema se encarga de se seleccionar los ítems que podrían interesarle a un determinado usuario.
\end{enumerate}

Ahora debemos encontrar un mecanismo para procesar toda la información que recopilemos previamente de los usuarios, para ello debemos reducir la dimensionalidad de nuestra matriz de votaciones, ya que esta puede ser muy grande y dispersa, a este proceso lo llamaremos \textit{Reducción de la Dimensionalidad}. 

Por lo que es necesario determinar una matriz que sea equivalente a la original y que sea mas concisa a la hora de brindar la información para realizar la recomendación. Esto permite hacer más eficiente el proceso pues solo tendremos que considerar las características principales en vez de analizar completamente toda nuestra extensa matriz original, además esto permite minimizar los problemas relacionados con la presencia de datos erróneos en nuestra recopilación.

\section{Descomposición en Valores Singulares(SVD)}

La factorización SVD consiste en expresar una matriz A de tamaño(n,d) como el producto de tres matrices:

$$ \underset{(n, d)}A \approx \underset{(n, n)}U \cdot \underset{(n, d)}\Sigma \cdot \underset{(d, d)} V^T  $$
Donde:
\begin{enumerate}
	\item[-] A es una matriz ortogonal de tamaño (n,n) que contiene a los vectores singulares izquierdos de A
	\item[-] $\sum_{}^{}$ es una matriz diagonal de tamaño (n,d)cuyos valores son los valores singulares de A ordenados en valor decreciente
	\item[-] V es una matriz de tamaño (n,d) que posee los valores singulares derechos de A
\end{enumerate}

Existe una propiedad 
aplicada a SVD enfocados en los sistemas 
de recomendación, esta consiste en que 
reduciendo el número de valores singulares 
de la matriz $\Sigma$ a los primeros k valores, se 
obtendrá una aproximación de la matriz 
original A, que permite ser reconstruida a 
partir de las versiones reducidas de las otras 
matrices cometiendo un cierto error, pero 
disminuyendo el tamaño. Es decir: 

$$ 
A_{n, m} \approx U_{n, k} \cdot \Sigma_{k, k} \cdot V^T_{k, m}  
$$

Esta propiedad es derivada del 
teorema de Eckart-Young que aborda 
la mejor aproximación a la matriz original 
A, obteniéndola poniendo a 0 los n valores 
singulares más pequeños, así se reducirán 
las matrices al número de valores singulares 
no nulos que tenga la matriz $\Sigma$. Esto resulta 
entonces en la transformación de gran cantidad 
de datos en su representación reducida, siendo 
por lo tanto una propiedad muy importante 
que permite reducir considerablemente el 
tiempo de cómputo de cálculo y de uso de 
memoria para las tres matrices.

\subsection{Filtrado Colaborativo}

El filtrado colaborativo es un método para hacer predicciones automáticas (filtrado) sobre los intereses de un usuario mediante la recopilación de las preferencias o gustos de información de muchos usuarios (colaborador). El Filtrado colaborativo se basa, en que si una persona A tiene la misma opinión que una persona B sobre un tema, A es más probable que tenga la misma opinión que B en otro tema diferente que la opinión que tendría una persona elegida azar. 

Dentro del filtrado colaborativo, es necesario el 
manejo de la matriz de votaciones de usuarios 
e ítems, por lo tanto, es posible considerar esta 
matriz como la base para aplicar SVD y obtener la 
factorización de matrices U, S y V. Luego, usando 
el teorema de Eckar-Youn, se puede reducir a k
dimensiones. La matriz de factorización sería:

$$ 
A_{usuarios, ítems} \approx U_{usuarios, k} \cdot \Sigma_{k, k} \cdot V^T_{k, ítems}
$$

Por otro lado, es posible simplificar 
el proceso de SVD obteniendo solo los 
factores de usuarios e ítems, esto es posible 
descomponiendo la matriz $\Sigma$ en dos matrices 
iguales, factores de usuarios $Ufac$ y factores de 
ítems $Ifac$ de esta manera:

$$
\Sigma_{k,k} = 
\left(
\begin{matrix}
\lambda_{1} &  & ...\\
& \lambda_{2}\\
... \\
& & \lambda_{k}
\end{matrix}
\right)
$$

$$
\Sigma_{k,k} = 
\left(
\begin{matrix}
\sqrt{\lambda_{1}} &  & ...\\
& \sqrt{\lambda_{2}}\\
... \\
& & \sqrt{\lambda_{k}}
\end{matrix}
\right) 
\cdot 
\left(
\begin{matrix}
\sqrt{\lambda_{1}} &  & ...\\
& \sqrt{\lambda_{2}}\\
... \\
& & \sqrt{\lambda_{k}}
\end{matrix}
\right)
$$

Nota : Los $\lambda_{i}$ son valores singulares de A.

Por tanto la matriz de votaciones se puede expresar como $Votos = Ufact\cdot Ifact$, donde :

$$
Ufact = U_{n, k} \cdot
\left(
\begin{matrix}
\sqrt{\lambda_{1}} &  & ...\\
& \sqrt{\lambda_{2}}\\
... \\
& & \sqrt{\lambda_{k}}
\end{matrix}
\right)
$$


$$
Ifact =
\left(\begin{matrix}
\sqrt{\lambda_{1}} &  & ...\\
& \sqrt{\lambda_{2}}\\
... \\
& & \sqrt{\lambda_{k}}
\end{matrix}
\right)
\cdot
V^T_{k, m}  
$$

Para hallar esta aproximación, la idea es buscar los 
factores de usuarios e ítems a partir de la matriz 
de votación abordándolo como un problema de 
regresión, donde se quieren encontrar los valores 
de las matrices de factores. Así, si se 
considera a $q_{i}$ al vector que represente los 
factores de un ítem y $p_{j}$ al vector que representa 
los factores del usuario se tendría que cumplir :

$$
A_{i, j} = q^T_{i} \cdot p_{j} \approx \sum_{n = 1}^{k} q_{i, n}\cdot p_{n, j}
$$

Ahora solo se ajustan los factores de 
aquellos usuarios que hayan definido un voto 
reduciendo considerablemente el sistema de 
ecuaciones y solucionando el problema de la 
dispersión, cometiendo un pequeño error $e_{i, j}$ = |$A_{i, j}$ - $q^T_{i} \cdot p_{j}$|. Debido a que el error absoluto es una 
función complicada de tratar, se usará el error 
cuadrado medio:

$$
(e_{i, j})^2 = (A_{i, j} - q^T_{i} \cdot p_{j})^2 \approx (A_{i, j} - \sum q_{i, n}\cdot p_{n, j})^2
$$

O sea, intentar encontrar el error mínimo para hacer la mejor aproximación de la siguiente forma:

$$
min_{p, q} = \sum (A_{i, j} - q_{i, n}\cdot p_{n, j})^2
$$

Pero es necesario además hacer una regularización para evitar el \textit{Outfitting}, así, es posible llegar a:

$$
(e_{i, j})^2 = (A_{i, j} - \sum q_{i, n}\cdot p_{n, j})^2 + \frac{\lambda}{2} \cdot (||q_{i}||^2 + ||p_{j}||^2) \space (1)
$$

Donde $\lambda$ es una constante de regularización.

Para encontrar el mínimo de la función ya detallada 
existen técnicas como la del descenso del gradiente que 
buscan ajustar las matrices de factores poco a 
poco de manera automática. Esto se realizará en la 
función (1), con el fin de encontrar el mínimo de 
la función, o acercarse lo suficiente a la solución. Esta técnica consiste en ir 
moviéndose poco a poco por la función hasta encontrar el mínimo evaluando, la función en un punto dado para 
luego moverse una distancia hacia otro punto; 
ahora se calcula la derivada en ese punto y 
se desplaza al lado opuesto de la derivada de 
la función que se quiere minimizar.

Aplicando descenso del gradiente en (1) se tiene:

$$
\frac{\delta(e_{i, j})^2}{\delta p_{i, n}} = -2q_{n, j}\cdot (A_{i, j} - \sum q_{i, n}\cdot p_{n, j}) + \lambda p_{i, n} \space (2)
$$

$$
\frac{\delta(e_{i, j})^2}{\delta q_{n, j}} = -2p_{i, n}\cdot (A_{i, j} - \sum q_{i, n}\cdot p_{n, j}) + \lambda q_{n, j} \space (3)
$$

Teniendo ya el gradiente, solo queda 
aplicar las reglas de actualización tanto para $p_{i, n}$
como para $q_{n, j}$ en (2), (3) con las constantes de 
regularización y aprendizaje. Así, la regla de 
actualización parte de $\sigma_{n + 1} = \sigma_{n} - \alpha \nabla f(x)$, donde $\sigma_{n + 1}$ es el nuevo valor en la matriz de votaciones, $\sigma_{n}$ su valor actual, $\alpha$ constante de aprendizaje y $\nabla f(x)$ es el gradiente ya obtenido. Dando como resultado :

Para $p_{i, n}$ :

$$
p_{i, n}^{'} = p_{i, n} + \alpha[2q_{n, j}\cdot (A_{i, j} - \sum q_{i, n}\cdot p_{n, j}) - \lambda p_{i, n}] \space (4)
$$

Para $p_{n, j}$ :

$$
q_{n, j}^{'} = q_{n, j} + \alpha[2p_{i, n}\cdot (A_{i, j} - \sum q_{i, n}\cdot p_{n, j}) - \lambda q_{n, j}] \space (5)
$$

El enfoque se plasma en las expresiones (4) y 
(5) que serán las que actualizarán los valores de 
la matriz de votaciones a partir de cada \textit{Ephochs}, 
tanto para los factores de usuarios, como para los 
factores de ítems.


\section{Creación del sistema de Recomendación en Python}

Para la creación del sistema de recomendación usaremos la bibliotecas suprise y pandas.

Necesitamos tener un dataset que contenga los items que, queremos recomendarle al usuario, el cual debe estar ubicado en la carpeta data con el nombre de items.csv y contar con las propiedades: itemId, title, genres en ese orden. Además necesitaremos otro dataset que cuente con los items que cada usuario tuvo la experiencia de apreciar y dar su respectiva recomendación, el mismo debe estar ubicado en la carpeta data con el nombre de ratings.csv y contar con las siguientes propiedades:
userId, itemId, rating, timestamp en ese orden.

Primero necesitamos cargar los datasets, esto lo haremos mediante pandas, luego luego debemos darle formato a nuestros datos para que sean reconocidos por el sistema, seguidamente procedemos a realizar el entrenamiento y el testing de los datos. Después de culminar el proceso anterior pasamos anterior entrenamos todos nuestros datos y creamos nuestra función de recomendación la cual recibe el id del usuario al cual queremos recomendar, el dataset con los datos, el modelo entrenado y la cantidad de items que queremos recomendarle al usuario.

El código fuente se encuentra disponible en \url{https://github.com/raudel25/SVD-SRI.git}




\end{document}

	

