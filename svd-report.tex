\documentclass{article}

\usepackage[utf8]{inputenc} 
	\usepackage[text={18cm,21cm},centering]{geometry}
	\usepackage{amsmath,amssymb,amsfonts,latexsym,cancel} 
	\usepackage[T1]{fontenc} % Comandos personales - especiales 
	\usepackage{titlesec} 
	\usepackage{hyperref}
	\newcommand{\sen}{\mathop{\rm sen}\nolimits} %seno 
	\newcommand{\arcsen}{\mathop{\rm arcsen}\nolimits} 
	\newcommand{\arcsec}{\mathop{\rm arcsec}\nolimits} 
	\newcommand{\R}{\mathbb{R}} \newcommand{\N}{\mathbb{N}} 
	\newcommand{\Z}{\mathbb{Z}} \def\max{\mathop{\mbox{\rm máx}}} % máximo \def\min{\mathop{\mbox{\rm mín}}} % mínimo

% Teoremas
%--------------------------------------------------------------------------
\newtheorem{thm}{Teorema}[section]
\newtheorem{cor}[thm]{Corolario}
\newtheorem{lem}[thm]{Lema}
\newtheorem{prop}[thm]{Proposición}
\newtheorem{defn}[thm]{Definición}
\newtheorem{rem}[thm]{Observación}

% De la misma forma se pueden definir comandos con argumentos. Por
% ejemplo, aquí definimos un comando para escribir el valor absoluto
% de algo más fácilmente.
%--------------------------------------------------------------------------
\newcommand{\abs}[1]{\left\vert#1\right\vert}

% Operadores.
% Los operadores nuevos deben definirse como tales para que aparezcan
% correctamente. Como ejemplo definimos en jacobiano:
%--------------------------------------------------------------------------
\DeclareMathOperator{\Jac}{Jac}

%--------------------------------------------------------------------------
\title{Sistemas de Recomendación basados en SVD}
\author{Daniel Abad Fundora C212\\
		Anabel Benítez González C211\\
		Raudel Alejandro Gómez Molina C211\\
		Alex Sierra Alcalá C211\\\\		
  \small SVD: Siempre venciendo desafios\\
  \small Matemática Numérica\\
  \small Facultad de Matemática y Computación 
}

\begin{document}
\maketitle

\section{Sistemas de Recomendación}

Los sistemas de recomendación son una herramienta que le ofrece a los usuarios una determinada ayuda a la hora de tomar decisiones, basado en la experiencia de otros usuarios. Un sistema de recomendación se puede definir como aquel sistema que tiene como principal tarea seleccionar ciertos objetos de acuerdo a los requerimientos del usuario. Estos sistemas son de gran utilidad cuando la cantidad de información ofrecida al usuario es muy superior a las capacidades de evaluación y exploración de este.\\\\
La creación de un sistema de recomendación cuenta con tres fases principales:
\begin{enumerate}
	\item[-] Captura de las preferencias y los gustos e intereses del usuario.
	\item[-] Extracción del conocimiento, aquí el sistema se encarga de interpretar la información que recopiló anteriormente para poder predecir los gustos y las preferencias del usuario.
	\item[-] A partir del contenido procesado anteriormente el sistema se encarga de se seleccionar los ítems que podrían interesarle a un determinado usuario.
\end{enumerate}

Ahora debemos encontrar un mecanismo para procesar toda la información que recopilemos previamente de los usuarios, para ello debemos reducir la dimensionalidad de nuestra matriz de votaciones, ya que esta puede ser muy grande y dispersa, a este proceso lo llamaremos \textit{Reducción de la Dimensionalidad}. 

Por lo que es necesario determinar una matriz que sea equivalente a la original y que sea mas concisa a la hora de brindar la información para realizar la recomendación. Esto permite hacer más eficiente el proceso pues solo tendremos que considerar las características principales en vez de analizar completamente toda nuestra extensa matriz original, además esto permite minimizar los problemas relacionados con la presencia de datos erróneos en nuestra recopilación.

\section{Descomposición en Valores Singulares(SVD)}

La factorización SVD consiste en expresar una matriz A de tamaño(n,d) como el producto de tres matrices:

$$ \underset{(n, d)}A \approx \underset{(n, n)}U \cdot \underset{(n, d)}\Sigma \cdot \underset{(d, d)} V^T  $$
Donde:
\begin{enumerate}
	\item[-] A es una matriz ortogonal de tamaño (mxm) que contiene a los vectores singulares izquierdos de A
	\item[-] $\sum_{}^{}$ es una matriz diagonal de tamaño (n,d)cuyos valores son los valores singulares de A ordenados en valor decreciente
	\item[-] V es una matriz de tamaño (n,d) que posee los valores singulares derechos de A
\end{enumerate}

\section{Creación del sistema de Recomendación en Python}

Para la creación del sistema de recomendación usaremos la bibliotecas suprise y pandas.

Necesitamos tener un dataset que contenga los items que, queremos recomendarle al usuario, el cual debe estar ubicado en la carpeta data con el nombre de items.csv y contar con las propiedades: itemId, title, genres en ese orden. Además necesitaremos otro dataset que cuente con los items que cada usuario tuvo la experiencia de apreciar y dar su respectiva recomendación, el mismo debe estar ubicado en la carpeta data con el nombre de ratings.csv y contar con las siguientes propiedades:
userId, itemId, rating, timestamp en ese orden.

Primero necesitamos cargar los datasets, esto lo haremos mediante pandas, luego luego debemos darle formato a nuestros datos para que sean reconocidos por el sistema, seguidamente procedemos a realizar el entrenamiento y el testing de los datos. Después de culminar el proceso anterior pasamos anterior entrenamos todos nuestros datos y creamos nuestra función de recomendación la cuak recive el id del usuario al cual queremos recomendar, el dataset con los datos, el modelo entrenado y la cantidad de items que queremos recomendarle al usuario.

El código fuente se encuentra disponible en \url{https://github.com/raudel25/SVD-SRI.git}




  

% Bibliografía.
%-----------------------------------------------------------------
\begin{thebibliography}{99}

\bibitem{Cd94} Autor, \emph{Título}, Revista/Editor, (año)

\end{thebibliography}

\end{document}
\documentclass{article}
	 
	


